\documentclass[a4paper,11pt]{article}
\usepackage{amssymb}
\usepackage{enumitem}
\usepackage{hyperref}

\usepackage{amssymb,amsmath,amsthm,latexsym,epsfig,euscript,multicol}
\usepackage[utf8x]{inputenc}
\usepackage{xcolor, bm}
\usepackage{listings}

\lstdefinelanguage{Python}
{
  morekeywords={from, import, def, return},
  comment=[l]{\#},
  morestring=[b]",
  alsodigit={-},
  alsoletter={&},
}
\lstdefinestyle{py}{
  language=Python,                 % the language of the code
  backgroundcolor=\color{white},   % choose the background color; you must add
  basicstyle=\ttfamily\scriptsize,        % the size of the fonts that are used for the code
  breakatwhitespace=false,         % sets if automatic breaks should only happen at whitespace
  breaklines=true,                 % sets automatic line breaking
  captionpos=b,                    % sets the caption-position to bottom
  commentstyle=\color{mygreen},    % comment style
  deletekeywords={...},            % if you want to delete keywords from the given language
  escapeinside={\%*}{*)},          % if you want to add LaTeX within your code
  extendedchars=true,              % lets you use non-ASCII characters; for 8-bits encodings only, does not work with UTF-8
  firstnumber=1,                % start line enumeration with line 1000
  frame=single,	                   % adds a frame around the code
  keepspaces=true,                 % keeps spaces in text, useful for keeping indentation of code (possibly needs columns=flexible)
  keywordstyle=\color{blue},       % keyword style
  morekeywords={*,...},            % if you want to add more keywords to the set
  numbers=left,                    % where to put the line-numbers; possible values are (none, left, right)
  numbersep=5pt,                   % how far the line-numbers are from the code
  numberstyle=\tiny\color{mygray}, % the style that is used for the line-numbers
  rulecolor=\color{black},         % if not set, the frame-color may be changed on line-breaks within not-black text (e.g. comments (green here))
  showspaces=false,                % show spaces everywhere adding particular underscores; it overrides 'showstringspaces'
  showstringspaces=false,          % underline spaces within strings only
  showtabs=false,                  % show tabs within strings adding particular underscores
  stepnumber=5,                    % the step between two line-numbers. If it's 1, each line will be numbered
  stringstyle=\color{mymauve},     % string literal style
  tabsize=4,	                   % sets default tabsize to 2 spaces
  title=\lstname                   % show the filename of files included with \lstinputlisting; also try caption instead of title
}
\lstset{style=py}


\definecolor{mygreen}{rgb}{0,0.6,0}
\definecolor{mygray}{rgb}{0.5,0.5,0.5}
\definecolor{mymauve}{rgb}{0.58,0,0.82}

% Caracteres especiales
\def\A{\mathbb{A}}
\def\C{\mathbb{C}}
\def \N{\mathbb{N}}
\def \P{\mathbb{P}}
\def \Q{\mathbb{Q}}
\def \R{\mathbb{R}}
\def \Z{\mathbb{Z}}
\def \sen{\textrm{sen}}

\def\Np{$\N$}
\def\Zp{$\Z$}
\def\Qp{$\Q$}
\def\Rp{$\R$}
\def\Cp{$\C$}

\def\bb{\bm{b}}
\def\bu{\bm{u}}
\def\bv{\bm{v}}
\def\bx{\bm{x}}
\def\bA{\bm{A}}
\def\bB{\bm{B}}
\def\bD{\bm{D}}
\def\bE{\bm{E}}
\def\bM{\bm{M}}
\def\bT{\bm{T}}


\def\K{\textrm{K}}
\def\V{\textrm{V}}
\def\S{\textrm{S}}

\def\degres{$^\circ$}

\newcount\todno
\def\no{\global\advance\todno by 1 \the\todno}

\topmargin-2cm \vsize 29.5cm \hsize 21cm
\setlength{\textwidth}{16.75cm}\setlength{\textheight}{23.5cm}
\setlength{\oddsidemargin}{0.0cm}
\setlength{\evensidemargin}{0.0cm}


\theoremstyle{definition}
\newtheorem{ejer}{Ejercicio}
\newcommand{\bej}{\begin{ejer}}
\newcommand{\fej}{\end{ejer}}

\begin{document}

\centerline{{\small Universidad de Buenos Aires - Facultad de Ciencias Exactas y Naturales - Ciencias de Datos}}

\vskip 0.2cm

\hrule

\vskip 0.2cm

 \centerline{{\bf\Large{\sc Laboratorio de Datos}}}

 \vskip 0.2cm

 \centerline{\ttfamily Primer Cuatrimestre 2024}

\vskip 0.2cm

 \hrule

 \bigskip
 \centerline{\bf Práctica N$^\circ$ 1: Nociones básicas de Python.}
 \bigskip

Como le dijo el Sr. Miyagi a Daniel, "\emph{\href{https://www.youtube.com/watch?v=tM-wKYKw0tI}{encerar... pulir}}". No se puede hacer ciencia de punta sin antes volver intuitivos los conceptos fundacionales de un lenguaje. Así que \emph{copien los comandos de esta guía a mano en una consola}, y traten de estimar qué van a devolver, antes de ejecutarlos.

Si ya instalaron el entorno de trabajo que sugiere el \lstinline|README.md|, pueden lanzar una consola adecuada ejecutando \lstinline|source venv/bin/activate && python|.

\begin{enumerate}
\item
Realizar las siguientes operaciones básicas en la consola
\begin{enumerate}
\item \begin{lstlisting}
2 + 2
a = 2
b = 5
a**3
16 %% 5  # probar con otros numeros para entender que significa
2 * a**2 + 0.5 * b + (a + b) / 2
\end{lstlisting}
\item Asignar el resultado anterior a la variable c, imprimir el contenido de c en la consola (corriendo c o print(c)). ¿Ven un [1] delante del valor de c? Esto indica que es un vector.
\end{enumerate}

\item Interpretar las siguientes operaciones lógicas y predecir el resultado antes de probar en la consola.
\begin{lstlisting}
a = True
b = False
a == b
a | b
a == (not b)
\end{lstlisting}

\item Antes de probar en la consola, piense que van a dar estas operaciones.
\begin{lstlisting}
a, b, c = 3, 4, 2
a > b
a <= b
a != b
a == b
not(a > b)
((c > a) or (10 * c > b)) and not(b / a > c)
\end{lstlisting}

\item Bonus: Abra la documentación de Python y lea sobre \href{https://docs.python.org/es/3/library/stdtypes.html\#tuples}{tuplas}.


\item \textbf{Listas.}
Las listas permiten guardar valores de distintos tipos en forma ordenada y acceder a los distintos elementos por su índice, comenzando desde 0.
\begin{lstlisting}
s = [1, 2, 3.0, "hola", 7 + 3]
s, s[0], s[1], s[-1]
\end{lstlisting}


\item \textbf{Vectores.}
Para trabajar con vectores en Python (y en general para todo tipo de operaciones matemáticas) vamos a usar el paquete \lstinline{numpy}. Para eso importamos primero la biblioteca \lstinline{numpy} y definimos vectores con el comando \lstinline{np.array}. Ejecutar el siguiente cxdigo y observar los resultados.

\begin{lstlisting}
import numpy as np
v = np.array([1,2,3])
w = np.array([1.2, 7, np.pi])
v, w, v + w
\end{lstlisting}

\item \textbf{La magia de Numpy} La biblioteca Numpy reproduce muchas funcionalidades de Matlab. La mayoría de las operaciones con vectores de Numpy se hacen coordenada a coordenada. Esto permite en muchos casos evitar usar ciclos o ciclos anidados y realizarlos con un solo comando.

Ejecutrar los siguientes comandos e interpretar los resultados. 

\begin{lstlisting}
np.set_printoptions(precision=2, suppress=True)
v = np.array([1,2,np.e,7])
w = np.array([1.2, np.pi, 4, 5])
for expr in [
    'v', 'w', 'v + w', 'v ** 2', 'v % 2', 'np.sum(v)', 'np.sqrt(w)', 'v > 3', 'w < 3.5'
]:
    print(f"{expr:11s} == {eval(expr)}")
\end{lstlisting}

\item Bonus: descrifrar qué magia hace la 
\href{https://docs.python.org/es/3.11/tutorial/inputoutput.html\#tut-f-strings}{f-string} que recibe como argumento \lstinline{print}. ¿Y eso de \href{https://docs.python.org/es/3.11/library/functions.html\#eval}{eval}?

\item Las operaciones lógicas \lstinline{or} y \lstinline{and} no se pueden aplicar a vectores. Debemos usar los símbolos \lstinline{|} (or) y \lstinline{&} (and).
\begin{lstlisting}
(v > 3) | (w < 3.5)
(v > 3) & (w < 3.5)
\end{lstlisting}

\item ¿Cómo se puede aplicar \lstinline{not} a un vector de variables booleanas ( $x \in \{\text{True, False}\}$ ? Pueden probar algunas ideas o buscar la respuesta en Internet.

\item Algunos comandos pueden dar resultados inesperados. Intenten adivinar cuál va a ser el resultado de cada comando.
\begin{lstlisting}
v = np.array([1,2,np.e,7])
w = np.array([1.2, np.pi, 4, 5])
z = np.array([0,1])
v * w
v+2
v+z
\end{lstlisting}

\item Explorar estas distintas formas de extraer información de un vector.
\begin{lstlisting}[language=Python]
v = np.array([1, 2, np.e, 7, 5])
v[0], v[1], v[-1]  # funcionara `v[-2]`?
v[[0, 3]]
v[0:3], v[0:1]
\end{lstlisting}

\item También podemos seleccionar los elementos que cumplan alguna propiedad.
\begin{lstlisting}
v = np.array([1, 2, np.e, 7, 5])
w = np.array([1, 0, 2, 5, 0])
v[v > 2]
v[w != 0]
\end{lstlisting}

\item \textbf{Matrices} Las matrices se definen en numpy como arrays de filas. Las operaciones usuales se realizan coordenada a coordenada al igual que con vectores.
\begin{lstlisting}
A = np.array([[3, 2, 2], [-1, 0, 1], [-2, 2, 4]])
B = np.array([[1, 0, 0], [0, 1, 0], [0, 0, 1]])
C = np.array([[0, 1, -1], [5, -2, 1]])
A + B
A * B
C**2  # bonus: pruebe con np.
\end{lstlisting}

\item El producto usual de matrices se realiza con el comando \lstinline{@}. \lstinline|A.T| es la transpuesta de A.
\begin{lstlisting}
A @ B
B @ C
B.shape, C.shape
B @ C.T
\end{lstlisting}

\end{enumerate}
\textbf{\large Funciones}

Las funciones son bloques de código organizado que se usan para realizar tares específicas. Reciben un input (un número o una variable, por ejemplo) y devuelven un output. Los inputs van entre paréntesis y separados por una coma, si hay más de uno. Muchas funciones están disponibles en la biblioteca estándar de Python, otras estás agrupadas en distintas bibliotecas, como \lstinline{numpy} que agrupa una gran cantidad de funciones matemáticas. El objetivo de estos ejercicios es familiarizarse con varias funciones básicas de Python.

\begin{enumerate}[resume]
\item Ejecutar estas operaciones en la consola para entender qué hacen las funciones de numpy:
\begin{lstlisting}
a = np.sqrt(2)
a
np.round(a)
np.round(a,2)
np.info(np.round)
np.info(np.ceil)
\end{lstlisting}

\item Muchas funciones de \lstinline{numpy} se pueden aplicar también en arrays:
\begin{lstlisting}
v = np.array([a, a**2, a**3, a**(.5)])
"v = ", v
"np.floor(v) = ", np.floor(v)
\end{lstlisting}

\item Explorar las funciones \lstinline{np.max()}, \lstinline{np.min()}, \lstinline{np.sum()}, \lstinline{np.mean()} y \lstinline{np.sort()} aplicadas al vector $v$ del ejercicio anterior. ¿Qué hace cada una?

\item Utilizando solo las operaciones y funciones vistas en los ejercicios anteriores, escribir códigos de una sola linea para las siguientes funciones matemáticas.
\begin{enumerate}
\item $\|v\|_2 = \sqrt{v_1^2 + v_2^2 + \dots + v_n^2} = \sqrt{\sum_{i=1}^n v_i^2}$
\item $\|v - w\|_2$
\item $\langle v, w\rangle = \sum_{i=1}^n v_i \cdot w_i$
\end{enumerate}

\item En Python podemos definir nuestras propias funciones utilizando \lstinline{def}. ¿Qué hace la siguiente función? ¿Qué resultados esperan al aplicar la función a los vectores $v_1$ y $v_2$?
\begin{lstlisting}
def todosPositivos(v):
  r = np.all(v>0)
  return(r)
  
v1 = np.array([3, 4])
v2 = np.array([3, 5, -1, 1])
todosPositivos(v1)
todosPositivos(v2)
\end{lstlisting}
  
\item Definir una función que calcule la norma-2 de un vector y verificarla en los vectores $v_1$ y $v_2$ del ejercicio anterior.

\item Usar las funciones \lstinline{np.argmin()} y \lstinline{np.argmax()} con el mismo vector de alturas. Interpretar qué hace cada una.

\item Reproducir estos usos de la función random.choice() e interpretar qué hace esta función.
\begin{lstlisting}
x = np.array(["cara", "ceca"])
s = random.choices(x, k = 10)
s
y = np.array([1,2,3,4,5,6])
t = random.choices(x, y = 10)
t
u = random.sample(x, y = 4)
\end{lstlisting}

Realicen otras pruebas para descubrir la diferencia entre \lstinline{random.choices()} y \lstinline{np.sample()}

\item {\textbf{El teorema central del límite.}} Este teorema asegura que si tiramos $n$ veces una moneda, el promedio de veces que sale cara tiende a $1/2$ cuando $n$ tiende a infinitio.
Utilizando un código de una línea, simular 10 lanzamientos de una moneda y calcular el promedio de veces que sale cara. Repetir para $n = 1000$ y $n = 100.000$.

Sugerencia: hay varias formas distintas de hacerlo, una posibilidad es usar el comando \lstinline{count} que puede aplicarse a listas.

\end{enumerate}

\textbf{\large Archivos de datos}

\begin{enumerate}[resume]
\item La biblioteca \lstinline{Pandas} nos permite trabajar fácilmente con archivos de datos.

\begin{enumerate}
\item Leer el archivo \lstinline{casos_coronavirus.csv}.
\item Graficar la curva de casos por día.
\item Graficar la curva de casos acumulados.
\item Definir $y$ como el logaritmo de la cantidad de casos acumulados y graficar $y$ en función de la cantidad de días transcurridos.
\item Estimar tomando dos valores la pendiente de la recta para los datos a partir del dia 30.
\end{enumerate}

Utilicen o modifiquen el siguiente código.

\begin{lstlisting}
import pandas as pd
datos = pd.read_csv("casos_coronavirus.csv")   # dataFrame
datos

# Convertimos los datos a np.array
datosNP = datos.to_numpy()
datosNP

x = np.linspace(1,96,96)
x

plt.plot(x, datosNP[:,2])

# Tomamos logaritmos para linealizar
y = np.log(np.float64(datosNP[:,2]))
y

plt.plot(x,y)
\end{lstlisting}

\end{enumerate}



\end{document}
