\documentclass[a4paper,11pt]{article}
\usepackage{amssymb, enumitem}
\setlist[enumerate,1]{label={\bfseries\arabic*.},ref={\bfseries{Ejercicio \arabic*}}}
\setlist[enumerate,2]{label={\bfseries(\alph*)},ref={\bfseries(\alph*)}}


\parindent 0cm
\usepackage{amssymb,amsmath,amsthm,latexsym,epsfig,euscript,multicol}
\usepackage{graphbox}

\usepackage[utf8x]{inputenc}
\usepackage{listings,xcolor,bm}


\definecolor{mygreen}{rgb}{0,0.6,0}
\definecolor{mygray}{rgb}{0.5,0.5,0.5}
\definecolor{mymauve}{rgb}{0.58,0,0.82}
\lstset{
  backgroundcolor=\color{white},   % choose the background color; you must add
  basicstyle=\ttfamily,      % the size of the fonts that are used for the code
  breakatwhitespace=true,         % sets if automatic breaks should only happen at whitespace
  breaklines=true,                 % sets automatic line breaking
  captionpos=b,                    % sets the caption-position to bottom
  commentstyle=\color{mygreen},    % comment style
  deletekeywords={...},            % if you want to delete keywords from the given language
  escapeinside={\%*}{*)},          % if you want to add LaTeX within your code
  extendedchars=true,              % lets you use non-ASCII characters; for 8-bits encodings only, does not work with UTF-8
  firstnumber=1,                % start line enumeration with line 1000
  frame=single,	                   % adds a frame around the code
  keepspaces=true,                 % keeps spaces in text, useful for keeping indentation of code (possibly needs columns=flexible)
  keywordstyle=\color{blue},       % keyword style
  language=Python,                 % the language of the code
  morekeywords={*,...},            % if you want to add more keywords to the set
  numbers=none,                    % where to put the line-numbers; possible values are (none, left, right)
  numbersep=5pt,                   % how far the line-numbers are from the code
  numberstyle=\tiny\color{mygray}, % the style that is used for the line-numbers
  rulecolor=\color{black},         % if not set, the frame-color may be changed on line-breaks within not-black text (e.g. comments (green here))
  showspaces=false,                % show spaces everywhere adding particular underscores; it overrides 'showstringspaces'
  showstringspaces=false,          % underline spaces within strings only
  showtabs=false,                  % show tabs within strings adding particular underscores
  stepnumber=5,                    % the step between two line-numbers. If it's 1, each line will be numbered
  stringstyle=\color{mymauve},     % string literal style
  tabsize=4,	                   % sets default tabsize to 2 spaces
  title=\lstname,                  % show the filename of files included with \lstinputlisting; also try caption instead of title
  belowskip=0em
}
% Caracteres especiales
\def\A{\mathbb{A}}
\def\C{\mathbb{C}}
\def \N{\mathbb{N}}
\def \P{\mathbb{P}}
\def \Q{\mathbb{Q}}
\def \R{\mathbb{R}}
\def \Z{\mathbb{Z}}
\def \sen{\textrm{sen}}

\def\Np{$\N$}
\def\Zp{$\Z$}
\def\Qp{$\Q$}
\def\Rp{$\R$}
\def\Cp{$\C$}

\def\bb{\bm{b}}
\def\bu{\bm{u}}
\def\bv{\bm{v}}
\def\bx{\bm{x}}
\def\bA{\bm{A}}
\def\bB{\bm{B}}
\def\bD{\bm{D}}
\def\bE{\bm{E}}
\def\bM{\bm{M}}
\def\bT{\bm{T}}


\def\K{\textrm{K}}
\def\V{\textrm{V}}
\def\S{\textrm{S}}

\def\degres{$^\circ$}

\newcount\todno
\def\no{\global\advance\todno by 1 \the\todno}

\topmargin-2cm \vsize 29.5cm \hsize 21cm
\setlength{\textwidth}{16.75cm}\setlength{\textheight}{23.5cm}
\setlength{\oddsidemargin}{0.0cm}
\setlength{\evensidemargin}{0.0cm}


\theoremstyle{definition}
\newtheorem{ejer}{Ejercicio}
\newcommand{\bej}{\begin{ejer}}
\newcommand{\fej}{\end{ejer}}

\begin{document}

\centerline{{\small Universidad de Buenos Aires - Facultad de Ciencias Exactas y Naturales - Ciencias de Datos}}

\vskip 0.2cm

\hrule

\vskip 0.2cm

 \centerline{{\bf\Large{\sc Laboratorio de Datos}}}

 \vskip 0.2cm

 \centerline{\ttfamily Primer Cuatrimestre 2024}

\vskip 0.2cm

 \hrule

 \bigskip
 \centerline{\bf Práctica N$^\circ$ 6: Operaciones con DataFrames y transformaciones de datos.}
 \bigskip


% \textbf{\large Visualizaci\'on}

Para estos ejercicios, usar el dataset \lstinline{penguins}. En la mayoría de los ejercicios se pide realizar varias transformaciones en un DataFrame. En estos ejercicios, ver primero la forma de hacerlo ``de cualquier manera'', utilizando una o varias instrucciones. Luego, si es posible, hacerlo mediante una sola instrucción encadenando las operaciones.

En todos los ejercicios, cuando se pide alguna condición, resolver el ejercicio implementando esa condición, no resolverlo a mano. Por ejemplo, si se pide una lista de los nombres de columnas de 14 caracteres, podemos hacer (ver \ref{ej:comprehension})

\lstinline{[col for col in penguins.columns if len(col) == 14]},

no tenemos que ponernos a contar a mano cu\'antas letras tiene cada nombre.


\begin{enumerate}

\item A partir del dataset \lstinline{penguins}, crear un subconjunto de datos que contenga sólo ping\"uinos de la isla Biscoe y que tengan un pico de 48 mm de largo o más.

Sugerencia: recordar que para realizar operaciones lógicas coordenada a coordenada con arrays de numpy podemos usar los símbolos $\&$ (and) y $\mid$ (or).

\item Crear otro dataset con la información de ping\"uinos Adelie machos que no sean de isla Biscoe.

\item Del dataset \lstinline{penguins} quedarse con todas las variables excepto \lstinline{year}, \lstinline{sex} y \lstinline{body_mass_g}.

Sugerencia: utilizar el método \lstinline{.drop()} de DataFrames.

\item \textbf{Restablecer índices.}
\begin{enumerate}
\item En el dataset \lstinline{penguins}, eliminar primero todas las filas con datos faltantes. ¿Qué sucede con los índices?
\item En el dataset sin datos faltantes, restablecer los índices mediante el comando \lstinline{reset_index()}.
\item ¿Cómo podemos hacer todo en un solo comando encadenando operaciones?
\end{enumerate}

\item \textbf{Renombrar columnas e \'indices.} Para renombrar columnas utilizamos

\lstinline{.rename(columns = ???)}

y para renombrar índices utilizamos

\lstinline{.rename(index = ???)}.

Realizar las siguientes operaciones en el dataset \lstinline{penguins}.
\begin{enumerate}
\item Renombrar la variable \lstinline{species} a \lstinline {especies}. En este caso debemos pasarle a \lstinline{columns} un diccionario: \lstinline|{'variable_original' : 'variable nueva'}|.
\item Renombrar en un solo \lstinline{rename} la variable \lstinline{flipper_length_mm} a \lstinline{aleta_mm} y la variable \lstinline{body_mass_g} a \lstinline{peso_g}.
\item Renombrar el \'indice 0 a ``cero''.
\item Pasar todos los nombres de variables a mayúsculas. Sugerencia: en lugar de un diccionario, podemos pasarle a \lstinline{columns} una función. En este caso, podemos usar la función \lstinline{str.upper}.
\item ¿Qué resultado esperan del siguiente comando?

\lstinline{penguins.rename(index = np.sqrt)}

\item ¿Cómo podemos sumarle uno a todos los índices de \lstinline{penguins}? Sugerencia: definir primero una función \lstinline{suma_uno} y utilizar esta función al hacer \lstinline{rename}.

\item En Python, al igual que en muchos lenguajes, podemos usar \emph{funciones lambda}, que nos permite crear funciones ``al vuelo''.
¿Qué resultado esperan del siguiente comando?

\lstinline{penguins.rename(index = lambda x : x * 2)}

\item ¿Cómo podemos usar una función lambda para renombrar todos los nombres de columnas a mayúsculas?
\end{enumerate}

\item \label{ej:comprehension} \textbf{df.columns.} También podemos renombrar columnas asignando una nueva lista de nombres mediante \lstinline{penguins.columns = ???}. En este caso, resulta útil definir \emph{listas por comprensión}. La sintaxis es \lstinline{[f(x) for x in LIST]}.

\begin{enumerate}
\item ?`Qué resultado esperan de ejecutar este comando?

\lstinline{[saludo + '!' for saludo in ['hola', 'chau']]}

\item En el dataset \lstinline{penguins}, convertir todos los nombres de variables a mayúsculas utilizando listas por comprensión.
\end{enumerate}

\item En el dataset \lstinline{penguins}, convertir solo los nombres de variables que empiezan con \lstinline{bill} a mayúsculas.

\textbf{Sugerencia:} Si queremos definir una lista for comprensión aplicando distintas funciones podemos crear una función partida utilizando \lstinline{if / else}. Por ejemplo, ¿cuál será la salida de la siguiente instrucción?

\lstinline{[x * 10 if x % 2 == 0 else x for x in [1,2,3,4,5,6]]}

\textbf{Nota:} la función que utilizamos para definir listas por comprensión es una función \emph{lambda} (pero no necesitamos escribir \lstinline{lambda}). Esta sintaxis para definir funciones partidas es parte de la sintaxis de las funciones lambda.

\item Utilizando una función lambda, agregar una nueva columna a la base de datos llamada \lstinline{peso_bin} que contenga:
\begin{itemize}
\item ``chico'' si la masa corporal es menos que 4000 gramos.
\item ``grande'' si la masa corporal es mayor que 4000 gramos.
\end{itemize}

\item
Crear un subconjunto de los datos de \lstinline{penguins} sólo con las obsevaciones de ping\"uinos machos con aletas (flipper) de más de 200 mm de largo y quedarse con todas las columnas que terminan con \lstinline{_mm}.

Sugerencias:
\begin{itemize}
\item Si queremos quedarnos con los elementos de una lista que cumplan una cierta propiedad, podemos hacer

\lstinline{[col for col in penguins.columns if ???]}

reemplazando \lstinline{???} por la condición.
\item Pueden utilizar el método \lstinline{endswith()} aplicado al string.
\end{itemize}

\item Empezando con \lstinline{penguins}, crear un dataframe con los siguientes dos requisitos:
\begin{enumerate}
\item contenga sólo con las observaciones de la isla Dream.
\item contenga solo las variables \lstinline{species} y todas las que empiecen con \lstinline{bill}.
\end{enumerate}


\item Empezando con \lstinline{penguins} realizar las siguientes operaciones:
\begin{enumerate}
\item Crear una nueva variable que tenga el peso en kg.
\item Convertir la variable \lstinline{island} a minúscula. Sugerencia: aplicar \lstinline{.str.upper()} a la columna.
\end{enumerate}

\item Agregar una columna a \lstinline{penguins} con la mediana de la masa corporal de los ping\"uinos de cada especie usando \lstinline{group_by()} y \lstinline{agg()}.

\item Empezando con \lstinline{penguins}, escribir una secuencia de operaciones que:
\begin{enumerate}
\item Excluya a los ping\"uinos observados en la isla Biscoe.
\item Sólo se quede con las variables que están entre \lstinline{species} y \lstinline{body_mass_g} inclusive.
\item Renombre la variable species a \lstinline{especie_pinguino}.
\item Agrupe los datos por la variable \lstinline{especie_pinguino}.
\item Calcule el valor medio de las variables que contienen el string ``length'', separando por la especie del ping\"uino, y llamando a las columnas como las originales pero agregando ``\_mean'' al final.
\end{enumerate}


\end{enumerate}
\textbf{Limpieza de datos.}
En los siguientes ejercicios, utilizamos el dataset \lstinline{macro_full_columns.csv}.

\begin{enumerate}[resume]
\item Queremos arreglar los nombres de algunas columnas y eliminar columnas inútiles.
\begin{enumerate}
\item Cargar el archivo en un DataFrame \lstinline{macroFull} utilizando la columna \lstinline{anio} como index.
\item Listar todos los nombres de columnas, y eliminar del DataFrame la columna \lstinline{Unnamed: 0}.
\item Observamos que algunas columnas terminan con el prefijo \lstinline{vari_Porc} y otras con el prefijo \lstinline{variPorc}. Cambiar el final de todas las columnas terminadas en \lstinline{vari_Porc} a \lstinline{variPorc}.
\item Modificar también todos los nombres de columnas terminados en \lstinline{_Per_Cap} a \lstinline{_perCap}.
\end{enumerate}

\item \textbf{Datos faltantes.}
\begin{enumerate}
\item ¿En qué columnas hay datos faltantes? Podemos usar \lstinline{df.isnull().any(axis = ???)}. ¿Cómo podemos generar una lista que tenga solamente los nombres de las columnas con datos faltantes?
\item ¿En qué años hay datos faltantes? Listar todos los años con datos faltantes.
\item Convertir todos los datos faltantes a 0.
\end{enumerate}

\item \textbf{Variables de oferta.}
\begin{enumerate}
\item Generar un DataFrame que contenga solo las variables que terminan con \lstinline{.oferta}. ¿Había datos faltantes en estas columnas?
\item Queremos explicar la variable \lstinline{PBI_a_precios_de_mercado.oferta} utilizando el resto de las variables de oferta. Crear un DataFrame \lstinline{X} que contenga todas las variables de oferta excepto la de PBI y una Series \lstinline{y} que contenga solo la variable de PBI.
\item Ajustar un modelo de regresión lineal ordinaria o Ridge y generar el vector de predicciones (hacerlo sobre el conjunto de todos los datos, no es necesario separar en entrenamiento y testeo).
\item Graficar, en un mismo gráfico, la variable respuesta original y la predicha en función del año. Sugerencia: prestar atención a los índices de cada variable.
\end{enumerate}

\end{enumerate}

\end{document}

\item Usar el método \lstinline{rename()} de DataFrames para cambiarle el nombre a la variable \lstinline{body_mass_g} y llamarla \lstinline{masa_corporal_g}.

\item Empezando con \lstinline{penguins}, contar cuántas observaciones hay por especie, isla y año.

\item Empezando con \lstinline{penguins}, quedarse sólo con los ping\"uinos de las especies Adelie y Gentoo. Luego contar cuántos hay por cada especie y sexo.


\item Empezando con \lstinline{penguins} crear una tabla resumen que contenga el largo mínimo y máximo de las aletas de los ping\"uinos de la especie Adelie, agrupados por isla.

\item Empezando con \lstinline{penguins}, agrupar los datos por especie y año, luego crear una tabla de resumen que contenga el ancho del pico (llamarla \lstinline{bill_depth_mean}) y el largo del pico (llamarla \lstinline{bill_length_mean}) para cada grupo

\item Empezando con \lstinline{penguins}, hacer una secuencia de operaciones que:
\begin{enumerate}
\item Agregue una nueva columna llamada \lstinline{bill_ratio} que sea el cociente entre el largo y el ancho del pico.
\item Quedarse sólo con las columnas \lstinline{species} y \lstinline{bill_ratio}.
\item Agrupar los datos por especie.
\item Crear una tabla de resumen que contenga el promedio de la variable \lstinline{bill_ratio} por especie y que el nombre de la columna en la tabla sea \lstinline{bill_ratio_mean}.
\end{enumerate}


\item Empezando con \lstinline{penguins},

\begin{enumerate}
\item Quedarse sólo con las observaciones correspondientes a ping\"uinos Chinstrap.
\item Luego, quedarse sólo con las variables \lstinline{flipper_length_mm} y \lstinline{body_mass_g}.
\item Agregar una nueva columna llamada \lstinline{fm_ratio} que contenga el cociente entre el largo de la aleta y el peso del ping\"uino.
\item Luego quedarse solo con las observaciones que no tienen NaN en ninguna columna (ayuda: \lstinline{dropna()})
\item Agregar otra columna llamada \lstinline{ratio_bin} que contenga la palabra ``alto'' si \lstinline{fm_ratio} es mayor o igual que $0.05$ y ``bajo'' si el cociente es menor que $0.05$.
\end{enumerate}
