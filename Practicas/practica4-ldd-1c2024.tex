\documentclass[a4paper,11pt]{article}
\usepackage{amssymb, enumitem}

\parindent 0cm
\usepackage{amssymb,amsmath,amsthm,latexsym,epsfig,euscript,multicol}
\usepackage[utf8x]{inputenc}
\usepackage{listings,xcolor,bm}


\definecolor{mygreen}{rgb}{0,0.6,0}
\definecolor{mygray}{rgb}{0.5,0.5,0.5}
\definecolor{mymauve}{rgb}{0.58,0,0.82}
\lstset{
  backgroundcolor=\color{white},   % choose the background color; you must add
  basicstyle=\small\ttfamily,      % the size of the fonts that are used for the code
  breakatwhitespace=false,         % sets if automatic breaks should only happen at whitespace
  breaklines=true,                 % sets automatic line breaking
  captionpos=b,                    % sets the caption-position to bottom
  commentstyle=\color{mygreen},    % comment style
  deletekeywords={...},            % if you want to delete keywords from the given language
  escapeinside={\%*}{*)},          % if you want to add LaTeX within your code
  extendedchars=true,              % lets you use non-ASCII characters; for 8-bits encodings only, does not work with UTF-8
  firstnumber=1,                % start line enumeration with line 1000
  frame=none,	                   % adds a frame around the code
  keepspaces=true,                 % keeps spaces in text, useful for keeping indentation of code (possibly needs columns=flexible)
  keywordstyle=\color{blue},       % keyword style
  language=Python,                 % the language of the code
  morekeywords={*,...},            % if you want to add more keywords to the set
  numbers=none,                    % where to put the line-numbers; possible values are (none, left, right)
  numbersep=5pt,                   % how far the line-numbers are from the code
  numberstyle=\tiny\color{mygray}, % the style that is used for the line-numbers
  rulecolor=\color{black},         % if not set, the frame-color may be changed on line-breaks within not-black text (e.g. comments (green here))
  showspaces=false,                % show spaces everywhere adding particular underscores; it overrides 'showstringspaces'
  showstringspaces=false,          % underline spaces within strings only
  showtabs=false,                  % show tabs within strings adding particular underscores
  stepnumber=5,                    % the step between two line-numbers. If it's 1, each line will be numbered
  stringstyle=\color{mymauve},     % string literal style
  tabsize=4,	                   % sets default tabsize to 2 spaces
  title=\lstname                   % show the filename of files included with \lstinputlisting; also try caption instead of title
}
% Caracteres especiales
\def\A{\mathbb{A}}
\def\C{\mathbb{C}}
\def \N{\mathbb{N}}
\def \P{\mathbb{P}}
\def \Q{\mathbb{Q}}
\def \R{\mathbb{R}}
\def \Z{\mathbb{Z}}
\def \sen{\textrm{sen}}

\def\Np{$\N$}
\def\Zp{$\Z$}
\def\Qp{$\Q$}
\def\Rp{$\R$}
\def\Cp{$\C$}

\def\bb{\bm{b}}
\def\bu{\bm{u}}
\def\bv{\bm{v}}
\def\bx{\bm{x}}
\def\bA{\bm{A}}
\def\bB{\bm{B}}
\def\bD{\bm{D}}
\def\bE{\bm{E}}
\def\bM{\bm{M}}
\def\bT{\bm{T}}


\def\K{\textrm{K}}
\def\V{\textrm{V}}
\def\S{\textrm{S}}

\def\degres{$^\circ$}

\newcount\todno
\def\no{\global\advance\todno by 1 \the\todno}

\topmargin-2cm \vsize 29.5cm \hsize 21cm
\setlength{\textwidth}{16.75cm}\setlength{\textheight}{23.5cm}
\setlength{\oddsidemargin}{0.0cm}
\setlength{\evensidemargin}{0.0cm}


\theoremstyle{definition}
\newtheorem{ejer}{Ejercicio}
\newcommand{\bej}{\begin{ejer}}
\newcommand{\fej}{\end{ejer}}

\begin{document}

\centerline{{\small Universidad de Buenos Aires - Facultad de Ciencias Exactas y Naturales - Ciencias de Datos}}

\vskip 0.2cm

\hrule

\vskip 0.2cm

 \centerline{{\bf\Large{\sc Laboratorio de Datos}}}

 \vskip 0.2cm

 \centerline{\ttfamily Primer Cuatrimestre 2024}

\vskip 0.2cm

 \hrule

 \bigskip
 \centerline{\bf Práctica N$^\circ$ 4: Regresión lineal.}
 \bigskip


% \textbf{\large Visualizaci\'on}

\begin{enumerate}[resume]
\item 
\begin{enumerate}
    \item Implementar una función que calcule la pendiente y la ordenada al origen de la recta de regresión lineal con las fórmulas vistas en clase:

     \[
     \begin{array}{rl}
          {\beta}_1 = & \dfrac{\displaystyle\sum_{i=1}^n(x_i - \bar{x})(y_i - \bar{y})}{\displaystyle\sum_{i=1}^n(x_i - \bar{x})^2}  \\[1em]
          {\beta}_0 = & \bar{y} - \hat{\beta}_1\bar{x}
     \end{array}
     \]

          donde:
     \[
     \begin{array}{rl}
          \bar{x} =& \dfrac{1}{n} \displaystyle\sum_{i=1}^n x_i  \\
          \bar{y} =& \dfrac{1}{n} \displaystyle\sum_{i=1}^n y_i
     \end{array}
     \]


    La función debe tomar como argumentos a \texttt{x} e \texttt{y}, que son \verb|pandas.Series| o \verb|numpy.array|, y devolver los valores de $\beta_0$ y $\beta_1$. 
    
    \textbf{Sugerencia:} recordar que dado un \texttt{pandas Series} se utiliza \texttt{.mean()} para calcular su promedio y recordar el uso de \texttt{np.sum}
    \vspace{0.2cm} 
\begin{lstlisting}
def coefs_rl(x, y):
    beta_1 = ???
    beta_0 = ???
    return beta_1, beta_0
\end{lstlisting}
    \item Con el dataset \texttt{gapminder}, utilizar la función implementada en el item anterior para realizar una regresión lineal entre los años y la expectiva de vida en Argentina. Comparar los coeficientes con los obtenidos por \texttt{scikit-learn}.
    \vspace{0.2cm} 
    
\begin{lstlisting}
datos = gapminder[???]
print(coefs_rl(datos[???], datos[???]))

modelo = linear_model.LinearRegression()
modelo.fit(datos[[???]], datos[[???]])
beta_1 = modelo.???
beta_0 = modelo.???
print(beta_1, beta_0)
\end{lstlisting}
\end{enumerate}

\item En este ejercicio trabajaremos con el dataset de inmuebles (\texttt{inmuebles.csv} en la página de la materia). El dataset contiene datos sobre inmuebles que están a la venta en cierta ciudad: su superficie en $m^2$, su precio en millones de pesos y la zona de la ciudad donde se encuentra. Recordar como cargar un dataset desde un \texttt{.csv} y visualizar sus primeras filas:
\vspace{0.2cm}
\begin{lstlisting}
datos = pd.read_csv('inmuebles.csv')
datos.head()
\end{lstlisting}

\begin{enumerate}
    \item Realizar un gráfico de dispersión (scatterplot) que muestre la relación entre la superficie y el precio de cada imueble.
    
    \item Realizar un gráfico de la regresión lineal entre ambas variables. El gráfico debe titularse ``Datos inmobiliarios" y la recta de Regresión Lineal debe tener una leyenda que diga ``Regresión".
    
    \item Calcular los coeficientes de la recta que mejor ajusta a los datos. Según el modelo, ¿qué podríamos interpretar sobre el costo del metro cuadrado en la ciudad?
    
    \item Para medir qué tan bien ajusta la recta a los datos, vamos a implementar dos funciones: una que calcule el error cuadrático medio (ECM) y otra que calcule el coeficiente de determinación $R^2$. Recordemos que:

    \[
    \begin{array}{rrl}
        \text{Error cuadrático medio:} & ECM =& \dfrac{1}{n}\displaystyle \sum_{i=1}^n (y_i - \hat{y}_i)^2 \\[2em]
         \text{Coeficiente de determinación:}& R^2 =& \dfrac{\displaystyle\sum_{i=1}^n (\hat{y}_i - \bar{y})^2}{\displaystyle\sum_{i=1}^n (y_i - \bar{y})^2} 
    \end{array}
    \]

    Para calcular ambas necesitamos los datos \texttt{x}, \texttt{y} y los coeficientes de la recta.

\begin{lstlisting}
def ecm(x, y, pendiente, o_origen):      
    return ???

def r_cuad(x, y, pendiente, o_origen):
    return ???
\end{lstlisting}

    \item Utilizando las funciones implementadas en el ítem anterior, calcular el ECM y el $R^2$ del ajuste realizado en el item b). ¿En qué unidades está cada medida? ¿Cómo podemos interpretarlas?

    \item Comparar los resultados obtenidos en el ítem anterior con los proporcionados por \texttt{r2\_score} y \texttt{mean\_squared\_error} de \texttt{scikit-learn}

    \item Mediante la confección de un boxplot, decidir en cuál de las zonas hay mayor variabilidad de precios. ¿Hay algún outlier?

    \item Para cada una de las zonas de la ciudad, calcular los coeficientes, el ECM y $R^2$ de la recta que mejor aproxima a los datos.
    
    \item Graficar los datos y el ajuste lineal de cada zona utilizando el método \lstinline{facet()} de \lstinline{Plot()} (recordar ejercicio 5.b de la práctica 3) ¿Cuál es el valor del metro cuadrado en cada zona? ¿Qué podemos concluir si comparamos estos valores con lo obtenido en el ítem c) ?

    \item Supongamos que queremos poner a la venta un inmueble de $105$ m$^2$. Sólo con esa información y teniendo en cuenta los items anteriores, ¿cuál sería el precio de refencia para la venta? Si sabemos además que el inmueble está en la Zona 2, ¿cambiaría en algo el valor calculado anteriormente? 

    \item Si me ofrecen un inmueble en la Zona 2 a un precio de 300, ¿qué tan barato o caro es respecto a su precio de referencia?

    \item \textit{Efecto de los outliers.} En este item trabajaremos con los datos de \texttt{inmuebles\_outliers.csv}, que tiene los mismos datos que \texttt{inmuebles.csv}, salvo cuatro que son outliers.
    \begin{enumerate}
        \item Realizar un boxplot que permita identificar en qué zona(s) se encuentran los outliers.
        \item Comparar los coeficientes del ajuste lineal de la(s) zona(s) afectada(s) con los obtenidos en el ítem h)
    \end{enumerate}
    

\end{enumerate}

\item En el archivo \texttt{bitcoin.csv} se encuentran datos de cotizacion de Bitcoin desde el 17/09/2014 hasta el 19/02/2022. Cargamos el dataset:
\begin{lstlisting}
btc = pd.read_csv('datos/bitcoin.csv')
btc.head()
\end{lstlisting}
Nos interesa analizar la evolución del precio de cierre (\textit{Close}) en periodo comprendido entre el 01/01/2021 y el 01/07/2021:
\begin{lstlisting}
# Nos aseguramos que pandas interprete la fecha correctamente
btc['Date'] = pd.to_datetime(btc['Date'], format='%Y-%m-%d') 

# Filtramos el dataset en el periodo de interes
btc_2021 = btc[(btc['Date']>"2021-01-01") & (btc['Date']<"2021-07-01")]
\end{lstlisting}

Visualizar el ajuste lineal para los datos del dataframe \verb |btc_2021|. En este caso, ¿resulta más conveniente un scatterplot o un gráfico de lineas para los datos? ¿Te resultaría útil utilizar esta recta para predecir el valor de BTC o para describir el cambio de su valor en este periodo?
    


\end{enumerate}

\end{document}

